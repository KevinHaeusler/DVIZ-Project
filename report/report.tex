%! Author = Kevin Häusler
%! Date = 08/11/2024

% Preamble
\documentclass[11pt,a4paper]{article}
% Packages
\usepackage[utf8]{inputenc}       % accents
\usepackage[T1]{fontenc}          % PS fonts
\usepackage{newtxtext,newtxmath}  % do not use CM fonts
\usepackage{amsmath}              % multi-line and other mathematical statements
\usepackage{setspace}             % setting the spacing between lines
\usepackage{graphicx}             % go far beyond what the graphics package
\usepackage[normalem]{ulem}       % various types of underlining
\usepackage{caption}              % rotating captions, sideways captions, etc.
\usepackage{float}                % tables and figures in the multi-column environment
\usepackage{subcaption}           % for subfigures and the like
\usepackage{longtable}            % tables that continue to the next page
\usepackage{multirow}             % tabular cells spanning multiple rows
\usepackage[table]{xcolor}        % driver-independent color extensions
\usepackage{lipsum}               % loren dummy text
\setlength{\marginparwidth}{2cm}  % todonotes' requirements
\usepackage{todonotes}            % todo's
\usepackage{csquotes}             % context sensitive quotation facilities
\usepackage[backend=biber,authordate]{biblatex-chicago}  % Chicago Manual of Style

%% document dimensions
\usepackage[a4paper, left=25mm,right=25mm,top=25mm,bottom=25mm,headheight=6mm,footskip=12mm]{geometry}
\setlength{\parindent}{0em}
\setlength{\parskip}{1ex}

%% headers & footers
\usepackage{lastpage}
\usepackage{fancyhdr}
\fancyhf{}                            % clear off all default fancyhdr headers and footers
\rhead{\small{\emph{\projtitle, \projauthor}}}
\rfoot{\small{\thepage\ / \pageref{LastPage}}}
\pagestyle{fancy}                     % apply the fancy header style
\renewcommand{\headrulewidth}{0.4pt}
\renewcommand{\footrulewidth}{0.4pt}

%% colors
\usepackage{color}
\definecolor{engineering}{rgb}{0.549, 0.176, 0.098}
\definecolor{cloudwhite}{cmyk}{0,0,0,0.025}

%% source-code listings
\usepackage{listings}
\lstset{ %
 language=C,                        % choose the language of the code
 basicstyle=\footnotesize\ttfamily,
 keywordstyle=\bfseries,
 numbers=left,                      % where to put the line-numbers
 numberstyle=\scriptsize\texttt,    % the size of the fonts that are used for the line-numbers
 stepnumber=1,                      % the step between two line-numbers. If it's 1 each line will be numbered
 numbersep=8pt,                     % how far the line-numbers are from the code
 frame=tb,
 float=htb,
 aboveskip=8mm,
 belowskip=4mm,
 backgroundcolor=\color{cloudwhite},
 showspaces=false,                  % show spaces adding particular underscores
 showstringspaces=false,            % underline spaces within strings
 showtabs=false,                    % show tabs within strings adding particular underscores
 tabsize=2,                         % sets default tabsize to 2 spaces
 captionpos=t,                      % sets the caption-position to top
 belowcaptionskip=12pt,             % space between caption and listing
 breaklines=true,                   % sets automatic line breaking
 breakatwhitespace=false,           % sets if automatic breaks should only happen at whitespace
 escapeinside={\%*}{*)},            % if you want to add a comment within your code
 morekeywords={*,var,template,new}  % if you want to add more keywords to the set
}

%% hyperreferences (HREF, URL)
\usepackage{hyperref}
\hypersetup{
    plainpages=false,
    pdfpagelayout=SinglePage,
    bookmarksopen=false,
    bookmarksnumbered=true,
    breaklinks=true,
    linktocpage,
    colorlinks=true,
    linkcolor=engineering,
    urlcolor=engineering,
    filecolor=engineering,
    citecolor=engineering,
    allcolors=engineering
}

%% path to the figures directory
\graphicspath{{figures/}}

%% bibliography file, must be in preamble
\addbibresource{bibliography.bib}

%% macros, to be updated as needed
\newcommand{\school}{HSLU: Lucerne University of Appplied Sciences and Arts}
\newcommand{\degree}{BSC Artificial Intelligence and Machine Learning}
\newcommand{\projtitle}{DVIZ Main Project}
\newcommand{\projauthor}{Kevin Häusler}
\newcommand{\tutor}{Prof.\ Elena Nazarenko, PhD}

%% my other macros, if needed
\newcommand{\windspt}{\textsf{WindsPT\/}}
\newcommand{\windscannerpt}{\emph{Windscanner.PT\/}}
\newcommand{\class}[1]{{\normalfont\sffamily #1\/}}
\newcommand{\svg}{\class{SVG}}

%% my environments for infos
\newenvironment{info}[1]{\vspace*{6mm}\color{blue}[ \textbf{INFO:} \begin{em} #1}
                        {\vspace*{3mm}\end{em} ]}
\newenvironment{infoopt}[1]{\vspace*{6mm}\color{blue}[ \textbf{INFO (elemento opcional):} \begin{em} #1}
                        {\vspace*{3mm}\end{em} ]}

%%------------------------------- document-------------------------------------

\begin{document}

%% preamble page numbers with arabic numerals
\pagenumbering{arabic}\setcounter{page}{1}

%%------------------------------- cover page ----------------------------------

\begin{titlepage}
\center

\vspace{-15mm}
{\large \textbf{\textsc{\school}}}\\

\vfill

{\Large \textbf{\projtitle}}\\[8mm]

{\Large \textbf{\projauthor}}\\

\vfill

%\includegraphics[width=52mm]{uporto-feup.pdf}

\vfill

{\large \degree}\\[8mm]
{\large \textbf{Tutor}: \tutor}\\[8mm]

%\renewcommand{\today}{15 de dezembro de 2023}
\today

\end{titlepage}

%%------------------------------- table of contents ---------------------------

%% redefine tableofcontents text, ONLY for PT
\renewcommand{\contentsname}{Table of Contents}

\tableofcontents
\newpage

%%------------------------------- table of work ---------------------------
\section{Work Table}

Due to me (Kevin Häusler) being the sole member of this project I have omitted the "done by" column.

\begin{table}[H]
\centering
\begin{tabular}{|l|l|l|}
\hline
\textbf{Date} & \textbf{Hours} & \textbf{Task description} \\ \hline
22.10.2024 &  0.1             &     Downloaded the json data from Shodan.io                       \\ \hline
 07.11.2024             & 0.5              &  Setup the project, convert the json data to xlsx, setup latex environment                         \\ \hline
 07.11.2024             &      1          &   Configure the initial streamlit setup                  \\ \hline
08.11.2024 & 1 & Start writing the report + setup Github \\ \hline
\end{tabular}
\end{table}

\newpage

%%------------------------------- About the  Project ---------------------------
\begin{about}
\section{About the Project}
\subsection{What is the project?}
The project is something akin to a cyber security analytic about the devices listed on shodan.io based on their location
being Rotkreuz. It is supposed to show insights about what is accessible and possibly insecure.

There are also dynamic parts where you can filter the data.
\subsection{Data}
The data is generated from shodan.io. I do have a lifetime membership that grants me enough credits to request and
download 10000 entries. I did filter the data to be only from Rotkreuz which resulted in a little bit over 9000
datapoints that I am using for this project.

\subsection{Tools}
Here is a list of the tools I am using:

PyCharm
Python 3.11
UV
Streamlit
Shodan CLI
Docker
Github
LaTeX

I have setup the project in pycharm with UV so it is easier to distribute, there is also a docker file in case you have
docker installed to easily run this project without having to create your own environment.

For the main library I have decided to use Streamlit, I have tried it out in the past but never for a real project so I
wanted to get more indepth experience with it. It has a great documentation and it lets me easily create dataframes and
charts for this project.

The whole report is also written in PyCharm with LaTeX and uploaded to the Github repository. It is written in the
report.tex file and compiled to report.pdf.

This setup is also useful in case I need to work from different machines.

\end{about}
\newpage

%%------------------------------- motivation ---------------------------
\begin{motivation}
\section{Motivation}
Reason why I selected this

\subsection{Cyber Security}
Test

\end{motivation}
\newpage

\end{document}